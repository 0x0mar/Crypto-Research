\documentclass[conference]{IEEEtran}
\usepackage{amsmath,amssymb,amsfonts}
\usepackage{url}
\usepackage{subfigure}
\usepackage{booktabs,threeparttable,multirow}

% new math operators
\DeclareMathOperator{\abs}{abs}

% todo command
\usepackage{marginnote}
\newcounter{todocnt}
\newcommand{\Sim}{\textsc{Simon}} 
\setcounter{todocnt}{0}
\newcommand{\todo}[1]{\stepcounter{todocnt}{\tt {[#1]}} \marginpar{{$\blacksquare$ \thetodocnt}}}  
\newcommand{\specialcell}[2][c]{%
  \begin{tabular}[#1]{@{}c@{}}#2\end{tabular}}

\hyphenation{op-tical net-works semi-conduc-tor}
\IEEEoverridecommandlockouts
\begin{document}

\title{Name of your Project goes here}


\author{\IEEEauthorblockN{Your names here
}
\IEEEauthorblockA{Worcester Polytechnic Institute, 
Worcester, MA 01609, USA\\
Email: \texttt{\{lpavel, jrmoore\}@wpi.edu}
}}
\maketitle

\begin{abstract}

  The Simon Cipher presented in ~\cite{Beaulieu_Simon} is a lightweight cipher created to better suit the embedded devices that are incrisingly being used in the society. Since physical access to a device offers the possibility of side channel attacks, the goal of this work is to privde an implementation of the Simon Cipher that is protected against them. The solution relies on achieving a constant leakage that cannot be used in order to discover the secret key.

\end{abstract}

\section{Motivation}

Once the internet takes over most of the devices people interact with, the concern of keeping information private becomes higher. Also small embedded devices need to have low costs and therefore do not have very high computational power. Thus small devices need ciphers that do not require very complex operations such as matrix multiplication in AES, and that also do not use very much memory.

Simon is one of the lightweight ciphers introduced by the NSA could be used by many small devices in the future. Since it has been proven that Simon is vulnerable to Differential Power Analysis, the motivation of our work is to make an implementation of the cipher that is immune to Side Channel Attacks by achieving a constant leakage in all operations.

The development kit used is a Stellaris LM3S8962 that has an ARM processor that runs at 50MHz. Because Side Channel Attacks are much easier to be done at a lower frequency, it was decided to use the board at a 5MHz clock rate. The motivation for choosing this platform was the popularity of ARM processors. Nowadays ARM processors are used in a very wide variety of products because of their very low power consumption.

Managed to cite ~\cite{BEPrince} . Awesome.
Managed to cite ~\cite{StandaertAdvances} . Awesome.
Managed to cite ~\cite{KocherDPA} . Awesome.
Managed to cite ~\cite{KocherIntroDPA} . Awesome.

List of goals:
\begin{itemize}
        \item achieve constant Hamming Weight and constant Hamming Distance between the states by different encodings.
        \item Perform side channel attacks on both implementations and check results.
\end{itemize}


We will stick to the following timeline:

\begin{itemize}
	\item 2/22: Project goals and outline defined
	\item 3/1 : Related work identified and described in report.
	\item 3/15: At least 1/3 of your anticipated work should be completed and documented
	\item 3/22: At least 2/3 of your anticipated work should be completed and documented
	\item 3/29: All of your anticipated work should be completed and documented
	\item 4/5:  Your results and outcomes are now also documented and included in the paper
	\item 4/12: Final submission of of complete paper for review
	\item 4/21: Reviews complete
	\item 4/26: Submission of final version addressing reviewers comments.
	\item 5/4 : Presentation of your project in class.
\end{itemize}

\section{Background}\label{sec:background}


\todo{You should find and describe related work early on. Know what other people have done.}

\section{Work Description}
Here you describe the work you have performed, problems you have solved and methods you have used. There is a fine balance between brevity and conciseness and ensuring that other people, if investing the time, would be able to reproduce your results given this description.



\section{Results}
\todo{here you will present and discuss your outcomes: implementation results or measurements or other project outcomes}

\section{Conclusion}
\todo{TBD last}



%\bibliographystyle{IEEETR}
\bibliographystyle{IEEEtran}
\bibliography{references}

\end{document}
